
%fecking magic
\geometry{mag=1600,truedimen}

%UofMN colors
\definecolor{UMNMaroon}{RGB}{122,0,25}
\definecolor{UMNLightGold}{RGB}{255,215,95}
\definecolor{UMNGold}{RGB}{255,204,51}
\definecolor{UMNStormy}{RGB}{64,77,91}
\definecolor{UMNSunny}{RGB}{0,149,182}
\definecolor{UMNLightGray}{RGB}{213,214,210}

\usepackage{textpos}
\usepackage{graphicx} % Allows including images
\usepackage{multirow}
\usepackage{epstopdf}
\usepackage{beamerthemesplit}
\usepackage{soul}
\usepackage{xifthen}
\usepackage{tikz}
\usepackage{listings}
\usepackage{siunitx}
\usepackage{xspace}
\definecolor{backcolour}{rgb}{0.95,0.95,0.92}

\lstdefinestyle{mystyle}{
    backgroundcolor=\color{UMNLightGray},
    commentstyle=\color{UMNStormy},
    keywordstyle=\color{UMNMaroon},
    numberstyle=\tiny\color{UMNMaroon},
    stringstyle=\color{UMNSunny},
    basicstyle=\ttfamily\footnotesize,
    emph={ldmx },
    emphstyle=\color{UMNMaroon},
    breakatwhitespace=false,
    breaklines=true,
    captionpos=b,
    keepspaces=true,
    numbers=left,
    numbersep=5pt,
    showspaces=false,
    showstringspaces=false,
    showtabs=false,
    tabsize=2
}

\lstset{style=mystyle}

%Import code into the listings environment from a file
% #1 options to give formatter
% #2 file path to code file
\newcommand{\codefile}[2]{%
    \lstinputlisting[#1]{#2}
}%

\newcommand{\code}[1]{%
  \tikz[baseline=(codebox.base)]{
    \node[inner sep=1pt,outer sep=0pt,draw=UMNStormy,line width=0.05mm,fill=UMNLightGray,rounded corners=0.03cm] (codebox) 
      {\textcolor{UMNSunny}{#1}};
  }%
}%

% a title within a slide's frame, useful for separating slide into sub sections
% #1 - text to style
\newcommand{\framesection}[1]{%
  \boldcol{UMNStormy}{#1}\hspace{1em}%
}%

\usepackage{bbding}
%Checkmark
\def\checkmark{{\color{UMNSunny}\Checkmark}}
%Crossmark
\def\crossmark{{\color{UMNMaroon}\XSolidBrush}}

%Done and Todo items 
\def\done{\item[$\color{UMNMaroon}\boxtimes$]}
\def\todo{\item[$\color{UMNMaroon}\square$]}

% Good and Bad items
\def\good{\item[\checkmark]}
\def\bad{\item[\crossmark]}

\usepackage{hyperref}

%Button that goes to a link when clicked
% #1 URL that you want to link to
% #2 Text in Button
\newcommand{\hlink}[2]{%
    \href{#1}{\beamergotobutton{#2}}
}%

%bolcol bold and color the input text
% #1 color
% #2 text
\newcommand{\boldcol}[2]{%
    {\color{#1}\textbf{#2}}\xspace
}%

%Only one plot on the slide
% #1 Slide Title
% #2 Slide Subtitle
% #3 Filepath to image
\newcommand{\oneplotslide}[3]{%
    \begin{frame}
        \frametitle{#1}
        \framesubtitle{#2}
        \begin{figure}
            \centering
            \includegraphics[width=\textwidth,height=0.8\textheight,keepaspectratio]{#3}
        \end{figure}
     \end{frame}
}%

%Put commands into four tiles
% #1 Slide Title
% #2 Slide Subtitle
% #3 Top Left Tile
% #4 Top Right Tile
% #5 Bottom Left Tile
% #6 Bottom Right Tile
\newcommand{\fourtileslide}[6]{%
    \begin{frame}
        \frametitle{#1}
        \framesubtitle{#2}
        \begin{figure}[h]
            \begin{tabular}{cc}
                #3 & #4 \\
                #5 & #6
            \end{tabular}
        \end{figure}
    \end{frame}
}%

%Tile four plots on the slide
% #1 Slide Title
% #2 Slide Subtitle
% #3 Filepath to top left image
% #4 Filepath to top right image
% #5 Filepath to bottom left image
% #6 Filepath to bottom right image
\newcommand{\fourplotslide}[6]{%
    \fourtileslide{#1}{#2}
        {\ifthenelse{\isempty{#3}}{%do nothing
            }{\includegraphics[height=0.4\textheight]{{#3}}}}%
        {\ifthenelse{\isempty{#4}}{%do nothing
            }{\includegraphics[height=0.4\textheight]{{#4}}}}%
        {\ifthenelse{\isempty{#5}}{%do nothing
            }{\includegraphics[height=0.4\textheight]{{#5}}}}%
        {\ifthenelse{\isempty{#6}}{%do nothing
            }{\includegraphics[height=0.4\textheight]{{#6}}}}%
}%

%Picture on left side with comments on right
%   #1 picture
%   #2 comments
\newenvironment{figurecomments}[1]{%
  \begin{columns}
    \begin{column}{0.6\textwidth}
      \begin{figure}
        \centering
        \includegraphics[height=0.7\textheight]{#1}
      \end{figure}
    \end{column}
    \begin{column}{0.4\textwidth}
      \footnotesize
}{%
    \end{column}
  \end{columns}
}%

%Environment for drawing on pictures using tikz
%   #1 Height to adjust image to (default: 0.4\textheight)
%   #2 Filepath to image
%Use like:
%   \begin{tikzimage}[height]{filepath}
%       ...\draw...( coordinates are in fractions of width/height of image )
%   \end{tikzimage}
\newenvironment{tikzimage}[2][0.4\textheight]{%
    \begin{tikzpicture}
        \node[anchor=south west, inner sep=0] (image) at (0,0)
            {\includegraphics[height=#1]{{#2}}};
        \begin{scope}[x={(image.south east)},y={(image.north west)}]
}{%
        \end{scope}
    \end{tikzpicture}
}%


%Section Title Card
\newcommand{\sectionframe}[1]{%
\begin{frame}
    \vfill
    \centering
    \begin{beamercolorbox}[sep=8pt,center,shadow=true,rounded=true]{title}
        \usebeamerfont{title}#1\par%
    \end{beamercolorbox}
    \vfill
\end{frame}
}%

%Wrap backup slides with these two commands to fool
% beamer into not including the backup slides in slide count
\newenvironment{backup}[1][Questions]{%
  \sectionframe{#1}
  \newcounter{finalframe}
  \setcounter{finalframe}{\value{framenumber}}
}{%
  \setcounter{framenumber}{\value{finalframe}}
}%
