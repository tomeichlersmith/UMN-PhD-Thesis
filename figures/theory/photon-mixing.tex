\begin{subfigure}[t]{0.48\textwidth}
    \centering
    \feynmandiagram [layered layout, horizontal=f1 to f2] {
    standard [particle={\(\gamma\)}] -- [photon] f1
    -- [fermion, half left, edge label=$\Phi$] f2
    -- [fermion, half left, edge label=$\Phi$] f1,
    f2 -- [photon] dark [particle={\(A'\)}],
    };
    \caption{Feynman diagram of how a massive field $\Phi$ could allow for a standard photon ($\gamma$)
        to mix with a dark photon ($A'$).}
    \label{fig:photon-mixing:with-field}
\end{subfigure}
\hfill
\begin{subfigure}[t]{0.48\textwidth}
    \centering
    \feynmandiagram [horizontal=sp to dp] {
    sm1 [particle={\(\ell^-\)}]
    -- [fermion] sp
    -- [fermion] sm2 [particle={\(\ell^+\)}],
    sp
    -- [photon, edge label=$\gamma$] mix [dot, label={\(\epsilon\)}]
    -- [photon, edge label=$A'$] dp,
    dm1 [particle={\(\overline{\chi}\)}]
    -- [fermion] dp
    -- [fermion] dm2 [particle={\(\chi\)}];
    };
    \caption{Feynman diagram showing how fermionic standard matter ($\ell^\pm$) and dark matter ($\chi$ and $\overline{\chi}$)
        could exchange particles via the mixing of their photons. The massive field $\Phi$ is encapsulated by the effective mixing
        parameter $\epsilon$.}
    \label{fig:photon-mixing:with-matter}
\end{subfigure}