% shared import of packages and command definitions for slides and text

\usepackage{silence} % silence warnings
% dont have tikz-feynman tell me LuaTex would be better on every vertex and line
\WarningFilter{tikz-feynman}{The key you tried}

\usepackage{epsfig} % Allows the inclusion of eps files
\usepackage{epic} % Enhanced picture mode
\usepackage{eepic} % Extensions for epic
\usepackage{url} % URL handling
\usepackage{longtable} % Tables that continue onto multiple pages
\usepackage{mathrsfs} % Support for \mathscr script
\usepackage{multirow} % Span rows in tables
\usepackage{bigstrut} % Space struts in tables up and down
\usepackage{amssymb} % AMS math symbols and helpers
\usepackage{graphicx} % Enhanced graphics support
\usepackage{setspace} % Adjust spacing in captions, single by default
\usepackage{xspace} % Automatically adjusting space after macros
\usepackage{amsmath} % \text, and other math formatting options
\usepackage{siunitx} % \num{} formatting and SI unit formatting
\usepackage{booktabs} % Enhanced tables with \toprule, etc.
\usepackage{hyperref} % Add clickable links to other parts of the document
\usepackage{subcaption}
\usepackage{graphicx} % Allows including images

\usepackage[noabbrev,capitalise]{cleveref} % Automatically determine \cref type

% Configure the cleveref package
\newcommand{\creflastconjunction}{, and } % Always use the serial comma

\usepackage[compat=1.1.0]{tikz-feynman} % for feynman diagrams
\tikzfeynmanset{
    warn luatex = {false}
}

\usepackage{hepunits}
% Configure the siunitx package
\sisetup{
    group-separator = {,}, % Use , to separate groups of digits, like 12,345
    list-final-separator = {, and } % Always use the serial comma in \SIlist
}

\newcommand{\fourgev}{\qty{4}{GeV}\xspace}
\newcommand{\eightgev}{\qty{8}{GeV}\xspace}
\newcommand{\ecal}{ECal}
\newcommand{\hcal}{HCal}

\usepackage{acronym}
\acrodef{dm}[DM]{Dark Matter}
\acrodef{sm}[SM]{Standard Model}
\acrodef{gr}[GR]{General Relativity}
\acrodef{cmb}[CMB]{Cosmic Microwave Background}
\acrodef{bao}[BAO]{Baryon Acoustic Oscillations}
\acrodef{hep}[HEP]{High Energy Physics}
\acrodef{wimp}[WIMP]{Weakly Interacting Massive Particle}
\acrodef{ecal}[ECal]{Electromagnetic Calorimeter}
\acrodef{hcal}[HCal]{Hadronic Calorimeter}
\acrodef{ldmx}[LDMX]{Light Dark Matter eXperiment}
\acrodef{hps}[HPS]{Heavy Photon Search experiment}
\acrodef{jlab}[JLab]{Thomas Jefferson National Accelerator Laboratory}
\acrodef{cebaf}[CEBAF]{Continuous Electron Beam Accelerator Facility}
\acrodef{kf}[KF]{Kalman Filter}
\acrodef{gbl}[GBL]{General Broken Lines}
\acrodef{svt}[SVT]{Silicon Vertex Tracker}
\acrodef{bdt}[BDT]{Boosted Decision Tree}
\acrodef{eot}[EoT]{Electrons on Target}
\acrodef{eat}[EaT]{\ac{ecal} as Target}
\acrodef{mm}[MM]{Missing Momentum}
\acrodef{simp}[SIMP]{Strongly-Interacting Massive Particle}
\acrodef{vps}[VPS]{Vertex Projection Significance}
\acrodef{fee}[FEE]{Full Energy Electron}
\acrodef{oim}[OIM]{Optimum Interval Method}

\def\minyzero{$y_{0,\min}$\xspace}
\def\maxyzeroerr{$\sigma_{y_0,\max}$\xspace}
\def\Psum{$P_\mathrm{sum}$\xspace}

%Environment for drawing on pictures using tikz
%   #1 Width to adjust image to (default: \textwidth)
%   #2 Filepath to image
%Use like:
%   \begin{tikzimage}[height]{filepath}
%       ...\draw...( coordinates are in fractions of width/height of image )
%   \end{tikzimage}
\newenvironment{tikzimage}[2][\textwidth]{%
    \begin{tikzpicture}
        \node[anchor=south west, inner sep=0] (image) at (0,0)
            {\includegraphics[width=#1]{{#2}}};
        \begin{scope}[x={(image.south east)},y={(image.north west)}]
}{%
        \end{scope}
    \end{tikzpicture}
}%


% HPS Diagrams
%   using tikz math library to name variables to helpful names
%   define command to draw boilerplate target/L1/L2
\usetikzlibrary{math,decorations.pathreplacing}

% colors for diagram
\definecolor{HPSTarget}{rgb}{0.55,0.52,0.54}
\definecolor{HPSTracker}{rgb}{0.0,0.5,0.5}
\definecolor{HPSEcal}{rgb}{0.8,0.5,0.2}

\tikzmath{%
  \openingslope = 0.15; % y changer per x change
  \sensorsep = 0.12; % separation between sensors in a layer
  \sensorlen = 1.5; % length of sensors in vertical direction
  \targetx = -3.0;
  \layeronex = -0.2;
  \layertwox = 2.0;
  \targethalfthick = 0.1;
  \ecalbarlen=1.5;
  \ecalbarwidth=0.3;
  % can't have a blank line in this block
  %
  % compute locations
  \layeroney = (\layeronex - \targetx) * \openingslope;
  \layertwoy = (\layertwox - \targetx) * \openingslope;
  \sensorhalfsep = \sensorsep / 2;
  \reflineendx = \layertwox+0.5;
  \reflineendy = (\reflineendx - \targetx) * \openingslope;
  % paths for truly displaced
  \eleslope=(2-0.1)/(2.5-\targetx-2);
  \eleyzero=\eleslope*(-2.0)+0.1;
  \elelayeronestereoy=\eleslope*(\layeronex - \sensorhalfsep+1.0)+0.1;
  \elelayeroneaxialy=\eleslope*(\layeronex + \sensorhalfsep+1.0)+0.1;
  \elelayertwostereoy=\eleslope*(\layertwox - \sensorhalfsep+1.0)+0.1;
  \elelayertwoaxialy=\eleslope*(\layertwox + \sensorhalfsep+1.0)+0.1;
  \posslope=(-1.8-0.1)/(2.5-\targetx-2);
  \posyzero=\posslope*(-2.0)+0.1;
  \poslayeronestereoy=\posslope*(\layeronex - \sensorhalfsep+1.0)+0.1;
  \poslayeroneaxialy=\posslope*(\layeronex + \sensorhalfsep+1.0)+0.1;
  \poslayertwostereoy=\posslope*(\layertwox - \sensorhalfsep+1.0)+0.1;
  \poslayertwoaxialy=\posslope*(\layertwox + \sensorhalfsep+1.0)+0.1;
  % paths for not displaced
  \ndeleslope = (2-0)/(2.5-\targetx-\targethalfthick);
  \ndelelayeronestereoy=\ndeleslope*(\layeronex - \sensorhalfsep - \targetx - \targethalfthick);
  \ndelelayeroneaxialy=\ndeleslope*(\layeronex + \sensorhalfsep - \targetx - \targethalfthick);
  \ndelelayertwostereoy=\ndeleslope*(\layertwox - \sensorhalfsep - \targetx - \targethalfthick);
  \ndelelayertwoaxialy=\ndeleslope*(\layertwox + \sensorhalfsep - \targetx - \targethalfthick);
  \ndposslope = (-1.8-0)/(2.5-\targetx-\targethalfthick);
  \ndposlayeronestereoy=\ndposslope*(\layeronex - \sensorhalfsep - \targetx - \targethalfthick);
  \ndposlayeroneaxialy=\ndposslope*(\layeronex + \sensorhalfsep - \targetx - \targethalfthick);
  \ndposlayertwostereoy=\ndposslope*(\layertwox - \sensorhalfsep - \targetx - \targethalfthick);
  \ndposlayertwoaxialy=\ndposslope*(\layertwox + \sensorhalfsep - \targetx - \targethalfthick);
  % paths for fake displaced
  \fdeleslope = (2-0)/(2.5-\targetx-\targethalfthick);
  \fdelelayeronestereoy=\fdeleslope*(\layeronex - \sensorhalfsep - \targetx - \targethalfthick);
  \fdelelayeroneaxialy=\fdeleslope*(\layeronex + \sensorhalfsep - \targetx - \targethalfthick);
  \fdelelayertwostereoy=\fdeleslope*(\layertwox - \sensorhalfsep - \targetx - \targethalfthick);
  \fdelelayertwoaxialy=\fdeleslope*(\layertwox + \sensorhalfsep - \targetx - \targethalfthick);
  \fdposx = \targetx+1.0;
  \fdposy = \fdeleslope*(\fdposx-\targetx-\targethalfthick);
  \fdposslope = (-1.8-\fdposy)/(2.5-\fdposx);
  \fdposyzero = \fdposslope*(\targetx-\fdposx)+\fdposy;
  \fdbadhitstereoy = \fdposslope*(\layeronex-\sensorhalfsep-\fdposx)+\fdposy;
  \fdbadhitaxialy = \fdposslope*(\layeronex+\sensorhalfsep-\fdposx)+\fdposy;
  \fdposlayertwostereoy=\fdposslope*(\layertwox - \sensorhalfsep - \fdposx)+\fdposy;
  \fdposlayertwoaxialy=\fdposslope*(\layertwox + \sensorhalfsep - \fdposx)+\fdposy;
  % true path is from center of target to L2 hits
  \fdpostruslope = (\fdposslope*(\layertwox-\fdposx)+\fdposy-0)/(\layertwox-\targetx-\targethalfthick);
  \fdtruhitstereoy = \fdpostruslope*(\layeronex-\sensorhalfsep-\targetx-\targethalfthick);
  \fdtruhitaxialy = \fdpostruslope*(\layeronex+\sensorhalfsep-\targetx-\targethalfthick);
}

\tikzset{
  vertex/.style={diamond,fill,inner sep=1.5pt},
  hit/.style={circle,fill,inner sep=1.5pt},
  miss/.style={circle,draw,fill=white,inner sep=1.0pt}
}

\newcommand{\drawhpsfirsttwolayers}{%
  \draw[HPSTracker, very thick]
    (\layeronex-\sensorhalfsep,-\layeroney-\sensorlen)
    -- (\layeronex-\sensorhalfsep,-\layeroney)
    (\layeronex+\sensorhalfsep,-\layeroney-\sensorlen)
    -- (\layeronex+\sensorhalfsep,-\layeroney)
    (\layeronex-\sensorhalfsep,\layeroney+\sensorlen)
    -- (\layeronex-\sensorhalfsep,\layeroney)
    (\layeronex+\sensorhalfsep,\layeroney+\sensorlen)
    -- (\layeronex+\sensorhalfsep,\layeroney);
  \draw[HPSTracker] (\layeronex,\layeroney+\sensorlen) node[above] {L1};
  
  \draw[HPSTracker, very thick]
    (\layertwox-\sensorhalfsep,-\layertwoy-\sensorlen)
    -- (\layertwox-\sensorhalfsep,-\layertwoy)
    (\layertwox+\sensorhalfsep,-\layertwoy-\sensorlen)
    -- (\layertwox+\sensorhalfsep,-\layertwoy)
    (\layertwox-\sensorhalfsep,\layertwoy+\sensorlen)
    -- (\layertwox-\sensorhalfsep,\layertwoy)
    (\layertwox+\sensorhalfsep,\layertwoy+\sensorlen)
    -- (\layertwox+\sensorhalfsep,\layertwoy);
  \draw[HPSTracker] (\layertwox,\layertwoy+\sensorlen) node[above] {L2};
  
  \draw[gray,dashed] (\targetx,0) -- (\reflineendx,\reflineendy);
  \draw[gray,dashed] (\targetx,0) -- (\reflineendx,0);
  \draw[gray,dashed] (\targetx,0) -- (\reflineendx,-\reflineendy);
  
  \fill[HPSTarget] (\targetx-\targethalfthick,+1) rectangle (\targetx+\targethalfthick,-1);
  \draw[HPSTarget] (\targetx,-1) node[below] {Target};
}

\newcommand{\drawhps}{%
  \foreach \layerX/\layerY in {
    0.0/0.15,
    2.0/0.30,
    4.0/0.45,
    6.0/0.60,
    8.0/0.75,
    10.0/0.90
  }{
    \draw[HPSTracker, very thick]
      (\layerX-\sensorhalfsep,-\layerY-\sensorlen)
      -- (\layerX-\sensorhalfsep,-\layerY)
      (\layerX+\sensorhalfsep,-\layerY-\sensorlen)
      -- (\layerX+\sensorhalfsep,-\layerY)
      (\layerX-\sensorhalfsep,\layerY+\sensorlen)
      -- (\layerX-\sensorhalfsep,\layerY)
      (\layerX+\sensorhalfsep,\layerY+\sensorlen)
      -- (\layerX+\sensorhalfsep,\layerY);
  }
  \draw[HPSTracker] (9.0,0.75+\sensorlen) node[above,font=\LARGE] {SVT};

  \foreach \barY in {
    0.9,
    1.2,
    1.5,
    1.8,
    2.1
  }{
    \draw[draw=HPSEcal,fill=HPSEcal!20!white]
      (12.0,\barY) rectangle (12+\ecalbarlen,\barY+\ecalbarwidth)
      (12.0,-\barY) rectangle (12+\ecalbarlen,-\barY-\ecalbarwidth)
    ;
  }
  \draw[HPSEcal] (12+\ecalbarlen/2,2.1+\ecalbarwidth) node[above,font=\LARGE] {ECal};

  \draw[gray,dashed] (\targetx,0) -- (12+\ecalbarlen+0.5,0);
  
  \fill[HPSTarget] (\targetx-\targethalfthick,+1) rectangle (\targetx+\targethalfthick,-1);
  \draw[HPSTarget] (\targetx,-1) node[below left,font=\LARGE] {Target};

  \node (decay) at (\targetx+1.0,0.05) [circle,fill,inner sep=1.5pt] {};
  \node (production) at (\targetx,0.0) [circle,fill,inner sep=1.5pt] {};

  \draw[black,very thick,->]
    (\targetx-1.5,0.0) node[anchor=south east,font=\Large] {\(e^-\)} -- (\targetx-0.2,0.0);
  \draw[black,very thick,->]
    (decay) -- (11.8,1.6) node[anchor=south east,font=\Large] {\(e^-\)};
  \draw[black,very thick,->]
    (decay) -- (11.8,-2.2) node[anchor=north east,font=\Large] {\(e^+\)};
  \draw[black,very thick,->]
    (production) -- (-1.0,-0.5-\sensorlen) node[anchor=east,font=\Large] {\(e^-\)};

  \draw[blue,thick,->] (production) -- (decay) node[midway,above,font=\Large] {\(V_D\)};
}
