\chapter{Conclusion}
\label{chapter:conclusion}

Both \ac{ldmx} and \ac{hps} are important and required experiments in our search
for the particle nature of \ac{dm}.
While neither has progressed far enough to exclude new parameter space
(or potentially discover \ac{dm} that was previously out of reach),
both experiments take novel approaches towards this search.
These approaches are required because the traditional tactics have not
yielded discovery and are unable to reach into these specific regions
of parameter space.

The \ac{ldmx} \ac{eat} analysis presented here is novel for \ac{ldmx}
not only due to the missing-energy search style, but also because we
took care to estimate the magnitude of potential systematic errors,
form a quantitative non-zero background prediction with its own
statistical uncertainty, and performed the statistical analysis
over several bins.
Previous \ac{ldmx} analyses focusing on zero-background searches,
while important and interesting in their own right, do not have the
flexibility necessary to be an analysis used with data collected
form a newly built apparatus.
The \ac{eat} analysis channel is such an analysis and it is well
prepared to be the first analysis of \ac{ldmx} data making new
physics conclusions about our universe.

The \ac{hps} \ac{simp} L1L2 analysis extended the \ac{simp} search
to a previously-unexplored reconstruction category, enabling us to
use a larger fraction of the data already collected by \ac{hps}.
This \ac{simp} search, while not excluding new phase space with the
current data set, shows promise.
We have been able to show that a displaced-vertex search with
the possibility of missing energy is possible.
Extending such a search to unexplored phase space, an area where no
other experiments have access, is simply awaiting the necessary
reconstruction studies in order to analyze later and larger data
sets already collected by the detector.

As newer and smaller experiments compared to other \ac{hep} experiments,
different excitements and difficulties arise.
While participating in the full breadth of a particle physics experiment
has taught me a lot and given me opportunity to carve my own path of learning,
the lack of larger support structures has revealed some specifically intricate
problems that \ac{ldmx} and \ac{hps} share.

\section{Simulation Validity}
The design and initial studies of any experiment requires some simulation
of the measurements it could perform.
After the experiment has been constructed and collects data, the simulation
is still incredibly useful for studying how \emph{known} processes present
themselves within the observations.
The collected data \emph{could} have a previously-unknown and rare process
within it, so we must be careful to avoid biasing our analysis in such a
way that would ruin the validity of our results.
A reasonable strategy is simply to avoid studying all of the data at once
and instead design and optimize the analysis on a particular subset of the
data.
This process is called ``blinding'' and was used in \cref{part:hps};
however, this also naturally means the analysis can easily be optimized
only for the special events within the subset and the events outside of
this subset are unaccounted for.

A valid simulation, one which correctly predicts the scale and shape
of the observable distributions, is a valuable tool that could partially
or completely replace this blinding procedure.
In this case, the simulated observations could be used to help design
and optimize the analysis; avoiding bias while also allowing the analysis
to explore the full breadth of potential events that could be seen.

While the ``valid simulation'' I describe here is not necessarily possible,
we can move forward towards it in order to make future analyses better.
One of the first studies that will be done with new \ac{ldmx} data will
be comparing how our simulated events compare to the events actually being
collected by the real detector.
Returning to these comparisons for \ac{hps} and its subsequent, larger
sets of data is being done and will be beneficial for those analyses as well
especially given the separation between data and simulation already being
observed within the pre-selection (for example, in \cref{fig:vertex-pre-selection}).

\section{Tracking Reconstruction}
Most of my work on tracking and the alignment of a tracking detector
was done within the context of \ac{hps}.
This particular tracking detector is somewhat of a ``trial by fire'' due
to its delicate, two-sided nature required by the specific physics for
which we are searching.
Nevertheless, this experience has given me a solid introduction to charged
particle tracking.

As discussed in \cref{chapter:hps:alignment}, charged particle tracking is
very complicated and it almost certainly requires familiarity with the
equally-complicated process of alignment.
Going forward, both \ac{ldmx} and \ac{hps} require robust and solid tracking
software in order to make the interpretations of such intricate algorithms simpler.
Without the backing of a larger collaboration to more completely test
and validate this complicated software, using a shared tracking framework
like ACTS\cite{acts} is necessary, a strategy starting to be employed in both
experiments.

\section{Reflection}
This dissertation work has shown me the full breadth of a particle physics experiment.
From initial design and simulation, to extensions and validations of said simulation,
to intricate considerations of potential systematic errors in the experiment,
to optimization of selections using various figures of merit,
to intentional analysis blinding to avoid statistical bias when developing an analysis,
to final unblinding after proposals and evaluations,
to post-mortem analysis of the leftover bits of data.

While I have not documented all of these steps within this document,
they all brought with them lessons for which I am grateful.
