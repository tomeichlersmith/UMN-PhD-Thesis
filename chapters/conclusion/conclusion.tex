\chapter{Conclusion}
\label{chapter:conclusion}

Both \ac{ldmx} and \ac{hps} are important and required experiments in our search
for the particle nature of \ac{dm}.
While neither has progressed far enough to exclude new parameter space
(or potentially discover \ac{dm} that was previously out of reach),
both experiments take novel approaches towards this search.
These approaches are required because the traditional tactics have not
yielded discovery and are unable to reach into the specific regions
of parameter space that are accessible by \ac{ldmx} and \ac{hps}.

As newer and smaller experiments compared to other \ac{hep} experiments,
different excitements and difficulties arise.

\section{Simulation Validity}

\section{Tracking Reconstruction}
Most of my work on tracking and the alignment of a tracking detector
was done within the context of \ac{hps}.
This particular tracking detector is somewhat a ``trial by fire'' due
to its delicate, two-sided nature required by the specific physics for
which we are searching.

\section{Reflection}
This dissertation work has shown me the full breadth of a particle physics experiment.
From initial design and simulation, to extensions and validations of said simulation,
to intricate considerations of potential systematic errors in the experiment,
to optimization of selections using various figures of merit,
to intentional analysis blinding to avoid statistical bias when developing an analysis,
to final unblinding after proposals and evaluations,
to post-mortem analysis of the leftover bits of data.

All of these small steps brought with them lessons.
Lessons for which I am grateful.
