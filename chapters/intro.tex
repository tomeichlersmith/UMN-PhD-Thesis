%%%%%%%%%%%%%%%%%%%%%%%%%%%%%%%%%%%%%%%%%%%%%%%%%%%%%%%%%%%%%%%%%%%%%%%%%%%%%%%
% intro.tex: Introduction to the thesis
%%%%%%%%%%%%%%%%%%%%%%%%%%%%%%%%%%%%%%%%%%%%%%%%%%%%%%%%%%%%%%%%%%%%%%%%%%%%%%%%
\chapter{Introduction}
\label{chapter:intro}
%%%%%%%%%%%%%%%%%%%%%%%%%%%%%%%%%%%%%%%%%%%%%%%%%%%%%%%%%%%%%%%%%%%%%%%%%%%%%%%%

The long and winding road of a dissertation is not always neatly packaged into a
template - this can easily be seen in my own experience. While focusing much
of my time on the technical aspects of data processing software for a proposed (yet
to be built) experiment, I found myself near the end of my journey without \emph{real}
data to analyze from which I can make \emph{real} conclusions about the physical world.
This motivated me to participate in another experiment -- extremely similar to the original
one -- sharing theoretical motivations, technological designs, and even people. This
experience of participating in two similar experiments has provided an abundant
field of learning opportunities (as well as roadblocks) for me and my journey as a new physicist.

Physics is a lazy science and this reductionist perspective is a main attraction for many physicists.
We\footnote{
    Here I use the ``royal we'' as representing my point of view of the culture within the field.
    It should in no way be construed as scientific or exact statements and does not represent every
    physicist's point of view.
} want to avoid memorizing as much as possible; therefore, motivating the idea of condensing sets
of observations into ``laws'' that can be represented in an even more compact mathematical form.
We make up these ``laws'' and the vocabulary surrounding them not to decieve but merely to make
communication about our observations and the experiments that make them easier. We sometimes
debate the origins of these laws and their true philosophical meaning, but often, on a day-to-day
basis, the background of them is unimportant. The interesting work comes from \emph{testing} them,
breaking them, and remaking them. For our purposes here, this is what a ``theory'' is: a package
of laws and their mathematical forms with which we can make predictions of the observations of
our experiments.

While I speak in the context of all physics, in reality, I am residing within a small corner.
Primarily concerned with individual particles and how they interact with one another, my field
can be described as investigating the foundations of the universe. Our experiments require giving
these particles comparatively large amounts of energy, contextualizing the name \gls{hep-full}.
Giving such small particles such large amounts of energy requires extrodinarily large and complex
apparatuses. \gls{hep} is filled with experiments many stories tall, collaborations consisting
of hundreds of institutions and thousands of people, and observations lasting years if not decades.
This grand scale is helpful to keep in mind when I speak of the experiments in this thesis -- only
a few institutions (less than 10 in each) and only dozens of collaborators. Contrary to many other
sciences and experimental methods, these experiments still last longer than a typical doctoral
student career. While I report on these experiments, I will not detail their full operation, design,
or capabilities. I will focus on the work that I was able to contribute; hopefully providing a
building block for these experiments and \gls{hep} to grow in the future, and give only the necessary
context for parts with which I am less familiar.

Before partitioning this thesis according to the two experiments in which I participated,
\cref{sec:sm} summarizes our laws and its representative theory,
\cref{chapter:dm} will explore the theoretical ground on which both of them rest which will
also provide necessary vocabulary for discussing these two experiments. \cref{part:ldmx} presents
the work of my initial years as a graduate student developing data processing and realistic simulation
software for a proposed experiment. This part includes a description of this proposed experiment in
\cref{chapter:ldmx:experiment}, \todo{insert rest of chapters}.
\cref{part:hps} describes the work of my last years showing
the difficulties of working with data taken in the real world (with all the complexities that implies).
Similar to \cref{part:ldmx}, this part describes the experimental setup in \cref{chapter:hps:experiment},
\todo{insert rest of chapters}.
We then conclude with \cref{chapter:conclusion} which returns back to this high-level-view
from which we can discuss what we have learned about physics via these two experiments.

\section{Standard Model of Particle Physics}
\label{sec:sm}
Short, quippy description of standard model. Mainly focused on giving reader the vocabulary necessary
to understand later sections on DM.

\begin{figure}
    \centering
    \includegraphics[width=\textwidth]{figures/intro/Standard_Model_of_Elementary_Particles.svg.png}
    \caption{
        The Standard Model of Particle Physics showing the twelve fermions and five bosons,
        their various properities (mass, charge, spin), labels (box and circle colors),
        and interactions (brown loops). Credit to Cush on wikipedia for
        providing this diagram freely accessible and usable for any purpose.
    }
    \label{fig:sm}
\end{figure}

%%%%%%%%%%%%%%%%%%%%%%%%%%%%%%%%%%%%%%%%%%%%%%%%%%%%%%%%%%%%%%%%%%%%%%%%%%%%%%%%
