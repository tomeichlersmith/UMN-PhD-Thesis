%%%%%%%%%%%%%%%%%%%%%%%%%%%%%%%%%%%%%%%%%%%%%%%%%%%%%%%%%%%%%%%%%%%%%%%%%%%%%%%
% intro.tex: Introduction to the thesis
%%%%%%%%%%%%%%%%%%%%%%%%%%%%%%%%%%%%%%%%%%%%%%%%%%%%%%%%%%%%%%%%%%%%%%%%%%%%%%%%
\chapter{Introduction}
\label{chapter:intro}
%%%%%%%%%%%%%%%%%%%%%%%%%%%%%%%%%%%%%%%%%%%%%%%%%%%%%%%%%%%%%%%%%%%%%%%%%%%%%%%%

The long and winding road of a dissertation is not always neatly packaged into a
template - this can easily be seen in my own experience. While focusing much
of my time on the technical aspects of data processing software for a proposed (yet
to be built) experiment, I found myself near the end of my journey without \emph{real}
data to analyze from which I can make \emph{real} conclusions about the physical world.
This motivated me to participate in another experiment -- extremely similar to the original
one -- sharing theoretical motivations, technological designs, and even people. This
experience of participating in two similar experiments has provided an abundant
field of learning opportunities (as well as roadblocks) for me and my journey as a new physicist.

Before partitioning this thesis according to the two experiments in which I participated,
\cref{chapter:dm} will explore the theoretical ground on which both of them rest which will
also provide necessary vocabulary for discussing these two experiments. \cref{part:ldmx} presents
the work of my initial years as a graduate student developing data processing and realistic simulation
software for a proposed experiment. This part includes a description of this proposed experiment in
\cref{chapter:ldmx:experiment}, \todo{insert rest of chapters}.
\cref{part:hps} describes the work of my last years showing
the difficulties of working with data taken in the real world (with all the complexities that implies).
Similar to \cref{part:ldmx}, this part describes the experimental setup in \cref{chapter:hps:experiment},
\todo{insert rest of chapters}.]
We then conclude with \cref{chapter:conclusion} which returns back to this high-level-view
from which we can discuss what we have learned about physics via these two experiments.

%%%%%%%%%%%%%%%%%%%%%%%%%%%%%%%%%%%%%%%%%%%%%%%%%%%%%%%%%%%%%%%%%%%%%%%%%%%%%%%%
