%%%%%%%%%%%%%%%%%%%%%%%%%%%%%%%%%%%%%%%%%%%%%%%%%%%%%%%%%%%%%%%%%%%%%%%%%%%%%%%%
%science.tex: Chapter on DM physics:
%%%%%%%%%%%%%%%%%%%%%%%%%%%%%%%%%%%%%%%%%%%%%%%%%%%%%%%%%%%%%%%%%%%%%%%%%%%%%%%%
\chapter{Dark Matter}
\label{chapter:dm}
%%%%%%%%%%%%%%%%%%%%%%%%%%%%%%%%%%%%%%%%%%%%%%%%%%%%%%%%%%%%%%%%%%%%%%%%%%%%%%%%

Many physicists throughout history have been puzzled by new phenomena, but the
confusion surrounding dark matter -- estimated to be roughly 85\% of the total
matter in the universe -- is surely one of the biggest puzzles. While many
astronomical observations at various scales have confirmed the existence of
dark matter, we have yet seen any observations of its particle interacting
with our detectors. This has ruled out many models for dark matter's particle
nature, but there are still many more available which can both explain the
current observations of dark matter gravitational and cosmological effects
while skirt the limitations defined by the \emph{lack} of observation in
particle experiments.

\section{Evidence for Dark Matter}

The first evidence physicists had for a unseen material floating throughout the universe
was the observation of galactic rotation curves \cite{rubin-rotationcurve-1980,rotationcurve-2000}.
These observations measure the speed of different stars within a galaxy and compare this speed to
the distance of that star from the center of the galaxy. We can calculate this relationship
using \gls{gr} \todo{rotation curve calculation citation} and the
observations differ drastically from this calculation. The stars within galaxies we've observed move
much faster than \gls{gr} would predict leaving us with two explanations: either \gls{gr} is not the correct theory
to use in this situation or there is more un-seen mass floating within the galaxy allowing these stars
to move faster without leaving the galactic orbit.

Other indirect measurements give us additional ways to access information about this odd phenomena.
Within the framework of \gls{gr}, since energy and mass actually warp the fabric of spacetime,
we expect to see light itself follow a bent path around massive objects - a phenomenum that is called
\gls{grav-lens} and is observed and well modeled by \gls{gr}'s predictions \cite{gravlensing-2004}.
The accuracy of \gls{gr} within this context - a mass and distance scale similar to the rotation curve oddities also observed -
put more requirements on any modified theory of gravity that could both explain the rotation curves and gravitational lensing.
Additionally, measurements on some of the largest scales and from the early universe
display signs of a certain mass density attributed to non-baryonic matter.
The \gls{cmb} and \gls{bao}\dots
\begin{itemize}
    \item Astronomy observations show lots of measurements
          \begin{itemize}
              \item galactic rotation curves \cite{rubin-rotationcurve-1980,rotationcurve-2000}
              \item gravitational lensing \cite{gravlensing-2004}
              \item CMB and Baryonic Acoustic Oscilations measure a certain mass density of non-baryonic matter
              \item Type1a Supernovae \cite{type1a-supernova-2010} and Big Bang Nucleosynthesis \cite{nucleosynthesis-1998}
              \item Probably not mini black holes \cite{constraints-primordial-black-holes-2021}
          \end{itemize}
    \item Can conclude pretty comfortably that DM exists
\end{itemize}

\section{Particle Nature of Dark Matter}
The theoretical possibilities explaining dark matter are broad \cite{darksectors-2016}
event when excluding ourselves to the assumption that the dark matter phenomenum is explained
by the existence of a dark matter particle. Here, I will refer to this particular dark matter
as \gls{dm} which, in general, needs to satisfy the following requirements.
\begin{itemize}
    \item \textbf{Dark} There has been no detection of these particles via the light observed with our telescopes;
          therefore, the dark matter has to not interact via the electromagnetic force.
    \item \textbf{Long Living} Measurements of \gls{dm}'s mass density and presence agree across time
          (from as early as the \gls{cmb} era), so \gls{dm} needs to have a long lifetime.
    \item \textbf{Observable Here} Measurements of \gls{dm} agree across different points in space,
          so it must be observable here on earth in some terresterial experiment.
    \item \textbf{Thermal Relic} Both \gls{dm} and standard matter have similar
          cosmological densities, so we expect some interaction (even a weak one) should connect their
          origins to the early universe allowing both to exist from the Big Bang.
    \item \textbf{Universal Density} Since we can indirectly measure a \gls{dm} mass density on
          cosmological scales, we impose the requirement that any \gls{dm} model needs to allow for
          this density.
\end{itemize}

\begin{itemize}
    \item What is DM's particle nature? Requirements:
          \begin{itemize}
              \item No EM observations $\rightarrow$ Dark
              \item Measurements across time agree $\rightarrow$ Long Living
              \item Measurements across space agree $\rightarrow$ can be observed here
              \item Similar density as SM $\rightarrow$ some interaction (even if weak) that connects their origins (thermal relic)
              \item Universal density must be allowed by model
          \end{itemize}
    \item Whole lot of options
          \begin{itemize}
              \item Thermal relic reminder (i.e. how mass gets connected to cross section due to universal density constraint) \cite{thermal-freezeout-diagram-1996}
              \item Massive field loop causes dark and standard photon to mix
              \item Details of model left to higher energies (EFT) $\rightarrow$ summarize it as a mixing parameter $\epsilon$ \cite{kinetic-mixing-1986}
              \item But what happens to the dark photon after that? Could stay in "dark sector" (invisible decay) or return to standard particles (visible decay)
              \item HPS focused on visible decays where the produced electron-positron pair's kinematics may rely on inner-workings of dark sector
              \item Categorize dark sector models ("Vanila", SIMPs, iDM, others...)
          \end{itemize}
    \item Pseudo-Dirac iDM
    \item Motivations for complicated model
          \begin{itemize}
              \item WIMPs pretty well constrained \cite{supercdms-2018,damic-2020,xenon1t-2018}
              \item Direct detection is hard with LDM \cite{ldmconstraints-2019}
              \item super-WIMPs at keV mass scale \cite{superwimps-2008}
              \item iDM is interesting for others as well \cite{darkseaquest-2018}
          \end{itemize}
\end{itemize}

%%%%%%%%%%%%%%%%%%%%%%%%%%%%%%%%%%%%%%%%%%%%%%%%%%%%%%%%%%%%%%%%%%%%%%%%%%%%%%%%
