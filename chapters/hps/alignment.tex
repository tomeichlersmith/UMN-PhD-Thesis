\chapter{Alignment}
\label{chapter:hps:alignment}

Particle detectors used to ``track'' particles (often called ``trackers'')
largely operate by precisely measuring the position of the particle at several
different points along its trajectory.
From these position measurements, we are able to extract other,
more physically helpful, measurements such as momentum and charge
(as long as the particles move through a magnetic field so their resulting
trajectory is curved in a well known shape).
We will not go into the process of converting these position measurements
into trajectories of individual particles (``tracks'')
or estimating physical variables of these tracks.
Instead, this serves as motivation:
without precise knowledge of position, we are unable to precisely estimate
momenta of particles -- significantly impacting the capabilities of the
experiment.

We need all of these position measurements to be relative
to the same reference point (i.e. in the same coordinate system);
thus, the process of measuring particle positions is practically
done by combining two position measurements.
\begin{enumerate}
	\item Position relative to individual pieces of the tracker (here called ``sensors'')
	\item ``Global'' position of these sensors
\end{enumerate}
More specifically, the software we use to deduce physical measurements from the data
output by a tracker consists of a detector model (which stores the positions of all
sensors in the same coordinate system) and a series of algorithms we can apply to
this detector model along with data from the sensors.

While there are different stragies to measure position within
an individual sensor of the tracker (e.g. pixel vs strip tracking detectors),
this measurement is not the limiting factor in the context of this chapter
and is often the measurement whose precision is easier to characterize
since the individual sensors can be designed, built, and tested individual
for a pre-determined precision.

The second position measurement -- the global position of the sensors themselves
-- is where a lot of complexity arises.
We construct the detector and thus know where we put the sensors,
but placing them into position with the precision we desire is extremely difficult
(not to mention their position could shift slightly once the magnet is turned on).

\begin{figure}
	\centering
	\includegraphics[width=0.5\textwidth]{one-offs/position-momentum-uncertainty/momentum-error-from-position-error.pdf}
	\caption{}
	\label{fig:momentum-error-from-position-error}
\end{figure}

\todo[example]{I'd like some back-of-the-envelope position error $\to$ momentum
	error as an example motivating the O(dozen micron) position precision we desire.}

While we can determine the position of these sensors within millimeter precision
during construction, gaining another few orders of magnitude in precision knowledge
of position is not feasible with physical-access tools.
\todo[phrasing]{not sure how to separate design position from construction survey
	from post-construction survey from in-situ alignment measurements with beam particles.}
This motivates another measurement ``tool'' that allows us to determine
(or more accurately, constrain) the position measurements of our sensors.

\section{Alignment Procedure}
The tool that we choose to use is the same particles that we will use the tracker
to study for our experimental goals.
The one change is instead of using the tracker to measure the properties of these particles,
we will select particles according to our knowledge of their properties so that we can use
them to constrain the position of the tracker sensors.
