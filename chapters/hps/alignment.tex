\chapter{Alignment}
\label{chapter:hps:alignment}

Particle detectors used to ``track'' particles (often called ``trackers'') largely operate by
precisely measuring the position of the particle at several different points along its trajectory.
From these position measurements, we are able to extract other, more physically helpful,
measurements such as momentum and charge (as long as the particles move through a magnetic field so
their resulting trajectory is curved in a well known shape).
This process typically proceeds through two steps.
In the first step, called ``pattern matching'', a set of hits is collected which could plausibly
be from the path of a charged particle.
In the second, ``track fitting'' step, the path and the momentum are determined from the
hits taking into account the magnetic field and hit resolution.
This serves as motivation: without precise knowledge of position,
we are unable to precisely estimate momenta of particles -- significantly
impacting the capabilities of the experiment.
Inaccurate hit positions result in hits being left off tracks, included
in tracks where they should not be, or negatively affecting the results
of the track fitting.

We need all of these position measurements to be relative to the same reference point (i.e. in the
same coordinate system); thus, the process of measuring particle positions is practically done by
combining two position measurements.
\begin{enumerate}
  \item Position relative to individual pieces of the tracker (here called ``sensors'')
  \item ``Global'' position of these sensors
\end{enumerate}
More specifically, the software we use to deduce physical measurements from the data
output by a tracker consists of a detector model (which stores the positions of all
sensors in the same coordinate system) and a series of algorithms we can apply to
this detector model along with data from the sensors.

While there are different stragies to measure position within an individual sensor of the tracker
(e.g. pixel vs strip tracking detectors), this measurement is not the limiting factor in the
context of this chapter and is often the measurement whose precision is easier to characterize
since the individual sensors can be designed, built, and tested individually for a pre-determined
precision relative to a well defined point, such as the center of the sensor.

The second position measurement -- the global position and orientation of the sensors themselves
-- is where a lot of complexity arises.
We construct the detector and thus know where we put the sensors, but placing
them into position with the precision we desire is extremely difficult (not to mention their
position could shift slightly once the magnet is turned on or due to thermal effects as
the detector temperature changes).

We can gain some understanding of the position precision required with the help of
\cref{fig:momentum-error-from-position-error}. There, we see that the momentum error roughly scales
by a factor of five relative to the position error and, as the position error goes up, the
momentum's distribution starts to distort away from a normal shape. In addition to the scaling of
the \emph{relative} error, the overall scale of these measurements are an important factor as well.

\begin{figure}
  \centering
  \includegraphics[width=0.5\textwidth]{one-offs/position-momentum-uncertainty/momentum-error-from-position-error.pdf}
  \caption{
    Momentum error derived from injecting a certain amount of position error
    (blue and orange) compared to normal distributions (red and purple)
    with known momentum error.
    These distributions are normalized to have unit integrals.
    The trials were done in a simplified experimental model with quantities
    similar to \ac{hps}: a \qty{2}{\giga\electronvolt} electron curving within a
    \qty{1}{\tesla} magnetic field, the sensors measuring the position were
    separated by \qty{5}{\centi\meter} but their separation were also allowed to
    deviate from the true value by the position error given.
  }
  \label{fig:momentum-error-from-position-error}
\end{figure}

To achieve a rough estimate of the scale of these position measurements, consider a
\qty{2}{\giga\electronvolt} electron curving within a \qty{1}{\tesla} magnetic field. This electron
would then have a \qty{6.6}{\meter} radius of curvature due to this magnetic field, meaning the
position measurements within sensors separated by \qty{5}{\centi\meter} would change relative to
one another by \emph{less than} \qty{1}{\milli\meter}. This absolute measurement of
$\sim\qty{100}{\micro\meter}$, combined with the necessary \qty{1}{\percent} relative error, means we
need to know the position measurements to the level of \qty{1}{\micro\meter}.

While we can determine the position of these sensors within millimeter precision during
construction, we need an ``in-situ'' method which allows us to determine positions
and angles during operation.
This motivates another measurement ``tool'' that allows us to determine (or more
accurately, constrain) the position measurements of our sensors.

\section{Alignment Procedure}
The tool that we use is the same particles that we will use the tracker to study for our
experimental goals. The one change is instead of using the tracker to measure the properties of
these particles, we will select particles according to our knowledge of their properties so that we
can use them to constrain the position of the tracker sensors.

This constraint is born out of concrete connection between the physical kinematics of the particles
and the positions they record in the tracker sensors.
We can imbue our knowledge of the particle trajectories in the magnetic field into the
mathematical form of the equations that are fit to the position measurements.
The mathematical fitting process estimates the best values of the starting position and 3D momentum
from a set of points given the assumed individual precision of these points.
This produces a prediction of where the particle ``should have'' passed through each sensor.
The difference (or ``error'') divided by the expected precision of the hits can be summed
in quadrature to provide the $\chi^2$ -- a measure of how close the fit and the measurements
are to one another.
If the sensors in the detector model were in slightly different
positions than where they are in the real detector, the data would present fits that are not as
good (quantitatively, have a higher $\chi^2$).
With this in mind, we can slightly ``move'' the sensors in the detector model in order to
improve the fits in the data (lowering the $\chi^2$).
In order to account for the fact that individual tracks may still vary due to our
imperfect knowledge of their trajectory shapes (e.g. the trajectory shape assumes only soft
interactions with the detector material but some tracks could have harder interactions distorting
their trajectory), we do this procedure over many tracks at once and focus on improving all of
their fits (minimizing the sum of their $\chi^2$).

This technique relies on the movements of the sensors to be small.
The process can easily fall into a so-called ``Weak Mode'' where our quantitative measure of
how good the fits are is achieving a locally minimum value, but the trajectories do not actually
align with data.
This motivates a multi-stage approach to alignment after a tracking detector has already been built.

\begin{enumerate}
  \item \textbf{Survey}: Directly measure positions of the sensors relative to other
        detector components as precisely as possible.
  \item \textbf{In-Situ Alignment}: While only allowing for certain types of movements,
        minimize the total $\chi^2$ of well-known tracks.
\end{enumerate}

Since physical measurements obtaining the precision we require are so difficult, the survey is
often split into multiple, cascading measurements as well.
For example, the positions of the sensors are precisely determined relative to their mounting
brackets which are then measured relative to the holding frame which is then measured relative
to the global coordinates within which the experiment operates.

The second stage where we perform small movements of the sensors in order to optimize the fit of
our data to our software detector model is also broken into many smaller stages.
The most common way is to only allow a certain set of physically motivated movements to occur
during certain alignment iterations.
This can allow us to ``walk'' towards a detector model that is more aligned
with the physical detector while maintaining physical understanding of the changes being made.

\ac{hps} uses Kalman-Filtering for track finding,
General Broken Lines \cite{gbl-2012} V03-01-00 for track fitting accounting for multiple-scattering,
and \textsc{MillepedeII} \cite{millepede-2011} V04-13-01 for determining alignment parameters
by minimization of the total $\chi^2$.
However, these tools are not enough to effectively align a tracking detector.
The \ac{hps} \ac{svt} is relatively small but still contains several hundred alignment parameters.
As mentioned above, the in-situ alignment procedure generally only allows for certain types
of movements (i.e. only a certain set of parameters are allowed to change) during single
alignment iterations, but then the natural question is what types of movements should be allowed.
While some movements can be well-motivated on physical grounds (for example, moving the sensor
along its own senstive direction), other movements may still be helpful and so many movement trails
are required.
This is where my specific work came in -- I (like several others) worked on testing different
sets of movements to see if they could help improve the detector alignment.
Along the way, I also helped develop a suite of helpful side-tools primarily focused
on visualizing the movements of the model sensors that result from alignment iterations.
This visualization is primarily helpful to see if the parameters resulting from the quantitative
minimization algorithm produce realistic movements of the detector model.

\section{Detector Visualization}

\cref{fig:example-det-vis-abs} demonstrates a quantitative visualization where
the positions and orientations of all the sensors within a specific coordinate system are shown.
This visualization is helpful for verifying that the detector model being used within the software
is placing the sensors in expected positions and orientations. For example, we observe the $y$ and
$z$ positions steadily increasing as the layer number of the sensor increases -- representing the
opening angle as designed within \ac{hps}. Additionally, we apply the designed ``flipping'' of
sensors (the factor of $\pm$ when calculating the angles with the axes)
when plotting to make sure that the sensors in the model are within $\pi/2$ of their true orientation.

\begin{figure}
  \centering
  \includegraphics[width=0.9\textwidth]{figures/hps/alignment/example-det-vis-abs.pdf}
  \caption{Plotting all six global coordinates completely specifying position and orientation
    of each of the sensors in the \ac{hps} tracking detector. Two different versions of the
    in-software detector model are plotted; however, on this scale, they overlap one another.}
  \label{fig:example-det-vis-abs}
\end{figure}

The differences of these coordinates between multiple versions of the detector model
provides deeper insight.
\cref{fig:example-det-vis-diff} demonstrates this strategy by subtracting
the coordinates of the original detector model (``Original'') from aligned detector model (``Aligned'').
Specifically, we can observe that the alignment iterations only allowed certain movements
(for example, no translation along or rotation away from the global $z$ axis was allowed).
The fact that these movements are very small shows that the physical survey which was done
to define the Original detector was close in position and orientation.

\begin{figure}
  \centering
  \includegraphics[width=0.9\textwidth]{figures/hps/alignment/example-det-vis-diff.pdf}
  \caption{Plotting all six global coordinates completely specifying position and
    orientation of each of the sensors in the \ac{hps} tracking detector relative to
    the ``Original'' detector from \cref{fig:example-det-vis-abs}. We can now observe
    the quantitative differences between the different versions.}
  \label{fig:example-det-vis-diff}
\end{figure}

\section{Results}
This movement of the sensor positions within our detector model is all well-and-good,
but we should actually be investigating how these movements affect the reconstructed
physical variables of tracks when using the updated detector model.
In general, the physical variables that should be investigated can depend on the
priorities of the experiment; for this analysis within \ac{hps} we wish to inspect
two key features.
\begin{itemize}
  \item Momentum Magnitude -- the total magnitude of the momentum of the reconstructed
    track is a key component in all analyses done with \ac{hps} data and thus is a
    priority to be well understood. Often, it is also common to inspect the momentum
    magnitude for different types of tracks so that it can be understood to in even
    finer detail. (i.e. $p$ vs $\tan\lambda$, $p$ vs $\phi$, $p$ separated by charge,
    etc.)
  \item Beamspot Position -- the extrapolation of the track back to the target is
    another helpful variable for analyses and can be checked directly since \ac{hps}
    has other means for estimating the beam spot on the target.
\end{itemize}
In these results, we look at a specific subset of the data collected within \ac{hps}
during 2016. This subset of the data uses a special trigger and analysis cuts that
focus on selecting events where a \ac{fee} is captured within the \ac{hps} detector volume.
An \ac{fee} is an electron that passed through the target with minimal interaction and
so its total energy is close to the known beam energy.
This dataset provides a fertile ground for studying the alignment since we have
knowledge about averages of two aspects of the tracks.
The momentum magnitude of these tracks should be close to the known
beam energy \qty{2.3}{\giga\electronvolt} and the transverse location of these tracks
at the target should be near the origin.
\cref{fig:example-align-d0} and \cref{fig:example-align-p} shows the comparison between
the Original and Aligned detectors whose models were shown above.
We can see that the Aligned detector achieves a tighter distribution in both of these
variables for both the top and bottom half of the \ac{svt} meaning that this detector
model is able to more precisely represent the true placement of the sensors when the
data was collected.

\begin{figure}
  \centering
  \begin{tabular}{cc}
    \includegraphics[width=0.49\textwidth]{figures/hps/alignment/d0_bottom.pdf}
    &
    \includegraphics[width=0.49\textwidth]{figures/hps/alignment/d0_top.pdf}
  \end{tabular}
  \caption{The transverse impact parameter of the tracks at the target
  $d_0 = \sqrt{x_0^2+y_0^2}$ which should be centered on zero to align with
  the known beamspot at the target.}
  \label{fig:example-align-d0}
\end{figure}

\begin{figure}
  \centering
  \begin{tabular}{cc}
    \includegraphics[width=0.49\textwidth]{figures/hps/alignment/p_bottom.pdf}
    &
    \includegraphics[width=0.49\textwidth]{figures/hps/alignment/p_top.pdf}
  \end{tabular}
  \caption{The magnitude of the track momenta which should be centered on
  the known beam value of \qty{2.3}{\giga\electronvolt}.}
  \label{fig:example-align-p}
\end{figure}

Unfortunately, due to the number of possible parameters that could be tuned,
the possibility of improving the alignment of the tracking detector is something
that can always be done.
Thus, while there is often a current ``best'' detector model, future analyzers
may need to re-process data with a later (and hopefully improved) model that
would be expected to be more accurate to the real detector that data was taken with.
The results shown here are just this -- a snapshot of the progress in aligning the
\ac{hps} \ac{svt}.
