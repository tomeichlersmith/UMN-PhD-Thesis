\chapter{Alignment}
\label{chapter:hps:alignment}

Particle detectors used to ``track'' particles (often called ``trackers'')
largely operate by precisely measuring the position of the particle at several
different points along its trajectory.
From these position measurements, we are able to extract other,
more physically helpful, measurements such as momentum and charge
(as long as the particles move through a magnetic field so their resulting
trajectory is curved in a well known shape).
We will not go into the process of converting these position measurements
into trajectories of individual particles (``tracks'')
or estimating physical variables of these tracks.
Instead, this serves as motivation:
without precise knowledge of position, we are unable to precisely estimate
momenta of particles -- significantly impacting the capabilities of the
experiment.

We need all of these position measurements to be relative
to the same reference point (i.e. in the same coordinate system);
thus, the process of measuring particle positions is practically
done by combining two position measurements.
\begin{enumerate}
	\item Position relative to individual pieces of the tracker.
	      While there are different stragies to do this measurement
	      (e.g. pixel vs strip tracking detectors), this measurement
	      is not the limiting factor in the context of this chapter.
	\item ``Global'' position of these individual pieces.
	      We construct the detector and thus know where we put the
	      individual pieces, but placing them into position with
	      the precision we desire is extremely difficult (not to
	      mention their position could shift slightly once the
	      magnet is turned on).
\end{enumerate}