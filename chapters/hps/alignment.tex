\chapter{Alignment}
\label{chapter:hps:alignment}

Particle detectors used to ``track'' particles (often called ``trackers'')
largely operate by precisely measuring the position of the particle at several
different points along its trajectory.
From these position measurements, we are able to extract other,
more physically helpful, measurements such as momentum and charge
(as long as the particles move through a magnetic field so their resulting
trajectory is curved in a well known shape).
We will not go into the process of converting these position measurements
into trajectories of individual particles (``tracks'')
or estimating physical variables of these tracks.
Instead, this serves as motivation:
without precise knowledge of position, we are unable to precisely estimate
momenta of particles -- significantly impacting the capabilities of the
experiment.

We need all of these position measurements to be relative
to the same reference point (i.e. in the same coordinate system);
thus, the process of measuring particle positions is practically
done by combining two position measurements.
\begin{enumerate}
	\item Position relative to individual pieces of the tracker (here called ``sensors'')
	\item ``Global'' position of these sensors
\end{enumerate}
More specifically, the software we use to deduce physical measurements from the data
output by a tracker consists of a detector model (which stores the positions of all
sensors in the same coordinate system) and a series of algorithms we can apply to
this detector model along with data from the sensors.

While there are different stragies to measure position within
an individual sensor of the tracker (e.g. pixel vs strip tracking detectors),
this measurement is not the limiting factor in the context of this chapter
and is often the measurement whose precision is easier to characterize
since the individual sensors can be designed, built, and tested individually
for a pre-determined precision.

The second position measurement -- the global position of the sensors themselves
-- is where a lot of complexity arises.
We construct the detector and thus know where we put the sensors,
but placing them into position with the precision we desire is extremely difficult
(not to mention their position could shift slightly once the magnet is turned on).

We can gain some understanding of the position precision required with the help of
\cref{fig:momentum-error-from-position-error}.
There, we see that the momentum error roughly scales by a factor of five relative to
the position error and, as the position error goes up,
the momentum's distribution starts to distort away from a normal shape.
In addition to the scaling of the \emph{relative} error, the overall scale of these
measurements are an important factor as well.

\begin{figure}
	\centering
	\includegraphics[width=0.5\textwidth]{one-offs/position-momentum-uncertainty/momentum-error-from-position-error.pdf}
	\caption{
		Momentum error derived from injecting a certain amount of position error
		(blue and orange) compared to normal distributions (red and purple)
		with known momentum error.
		These distributions are normalized to have unit integrals.
		The trials were done in a simplified experimental model with quantities
		similar to \ac{hps}: a \qty{2}{\giga\electronvolt} electron curving within a
		\qty{1}{\tesla} magnetic field, the sensors measuring the position were
		separated by \qty{5}{\centi\meter} but their separation were also allowed to
		deviate from the true value by the position error given.
	}
	\label{fig:momentum-error-from-position-error}
\end{figure}

Just to achieve a rough estimate of the scale of these position measurements,
consider a \qty{2}{\giga\electronvolt} electron curving within a \qty{1}{\tesla}
magnetic field.
This electron would then have a \qty{6.6}{\meter} radius of curvature
due to this magnetic field, meaning the position measurements within
sensors separated by \qty{5}{\centi\meter} would change relative to one another
by \emph{less than} \qty{1}{\milli\meter}.
This absolute measurement of $\sim\qty{100}{\micro\meter}$ combined with the
necessary \qty{1}{\percent} relative error means we need to know the position
measurements to the level of \qty{1}{\micro\meter}.

While we can determine the position of these sensors within millimeter precision
during construction, gaining another few orders of magnitude in precision knowledge
of position is not feasible with physical-access tools.
\todo[phrasing]{not sure how to separate design position from construction survey
	from post-construction survey from in-situ alignment measurements with beam particles.}
This motivates another measurement ``tool'' that allows us to determine
(or more accurately, constrain) the position measurements of our sensors.

\section{Alignment Procedure}
The tool that we choose to use is the same particles that we will use the tracker
to study for our experimental goals.
The one change is instead of using the tracker to measure the properties of these particles,
we will select particles according to our knowledge of their properties so that we can use
them to constrain the position of the tracker sensors.

This constraint is born out of concrete connection between the physical
kinematics of the particles and the positions they record in the tracker sensors.
We can imbue our knowledge of the shape of particle trajectories in magnetic field
into the mathematical form of the equations that are fit to the position measurements.
If the sensors in the detector model were in slightly different positions
that where they are in the real detector, the data would present fits that are
not as good (quantitatively, have a higher $\chi^2$). With this in mind, we
can slightly ``move'' the sensors in the detector model with focus once
improving the fits in the data (lowering the $\chi^2$). In order to account
for the fact that individual tracks may still vary due to our imperfect knowledge
of their trajectory shapes (e.g. the trajectory shape assumes only soft interactions
with the detector material but some tracks could have harder interactions distorting
their trajectory), we do this procedure over many tracks at once and focus on improving
all of their fits (minimizing the sum of their $\chi^2$).

This technique relies on the movements of the sensors to be small.
The process can easily fall into a so-called ``Weak Mode'' where our quantitative
measure of how good the fits are is achieving a locally minimum value, but
the trajectories do not actually align with data. This motivates a multi-stage
approach to alignment after a tracking detector has already been built.

\begin{enumerate}
	\item \textbf{Survey}: Directly measure positions of the sensors relative to other
	      detector components as precisely as possible.
	\item \textbf{In-Situ Alignment}: While only allowing for certain types of movements,
	      minimize the total $\chi^2$ of well-known tracks.
\end{enumerate}

Since physical measurements obtaining the precision we require are so difficult,
the survey is often split into multiple, cascading measurements as well
For example, the positions of the sensors are precisely determined relative to their
mounting brackets which are then measured relative to the holding frame
which is then measured relative to the global coordinates within which the
experiment operates.

The second stage where we perform small movements of the sensors in order to optimize
the fit of our data to our software detector model is also broken into many smaller
stages. The most common way is to only allow a certain set of physically motivated
movements to occur during certain alignment iterations. This can allow us to ``walk''
towards a detector model that is more aligned with the physical detector while
maintaining physical understanding of the changes being made.

\section{Detector Visualization}

In some sense, both of these stages inform the detector model and, as such,
a critical piece of alignment is a quantitative visualization of this detector
model and how it is being modified. \cref{fig:example-det-vis-abs} demonstrates a
quantitative visualization used within \ac{hps} where the positions and orientations of 
all the sensors within a specific coordinate system are shown.
This visualization is helpful for verifying that the detector model being used
within the software is placing the sensors in expected positions and orientations.
For example, we observe the $y$ and $z$ positions steadily increasing as the layer
number of the sensor increases -- representing the opening angle as designed within
\ac{hps}. Additionally, we see many of the sensors ``flipped'' along one or more
directions (the angle is $\pi \approx \qty{3141}{\milli\radian}$) which can be
double checked against design and construction.

\begin{figure}
	\centering
	\includegraphics[width=\textwidth]{figures/hps/alignment/example-det-vis-abs.pdf}
	\todo[figure]{patch axis angles column label in absolute det vis}
	\caption{Plotting all six global coordinates completely specifying position and orientation
	of each of the sensors in the \ac{hps} tracking detector. Three different versions of the
	in-software detector model are plotted; however, on this scale, they all overlap one another.}
	\label{fig:example-det-vis-abs}
\end{figure}

Of more specific help is looking at the differences of these coordinates between
multiple versions of the detector model. \cref{fig:example-det-vis-diff} demonstrates
this strategy by subtracting the coordinates of the original detector model (``OG'')
from the subsequent attempts to further align it. Specifically, we can observe that
the alignment iterations only allowing translations along the vertical did in fact
only change in the global $y$ direction. Similarly, when we allowed for rotations as
well as these translations, some angles in the global frame changed (orange).
In both cases, neither $x$ nor $z$ coordinates changed relative to the original model.

\begin{figure}
	\centering
	\includegraphics[width=\textwidth]{figures/hps/alignment/example-det-vis-diff.pdf}
	\caption{Plotting all six global coordinates completely specifying position and
	orientation of each of the sensors in the \ac{hps} tracking detector relative to
	the base detector ``OG'' from \cref{fig:example-det-vis-abs}. We can now observe
	the quantitative differences between the different versions.}
	\label{fig:example-det-vis-diff}
\end{figure}

\section{Results}
\todo[find]{find some figures of unaligned vs aligned physical quantities to show
that this process is necessary and helpful.}