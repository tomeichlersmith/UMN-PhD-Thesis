%%%%%%%%%%%%%%%%%%%%%%%%%%%%%%%%%%%%%%%%%%%%%%%%%%%%%%%%%%%%%%%%%%%%%%%%%%%%%%%%
% simulation.tex: Chapter on MC production:
%%%%%%%%%%%%%%%%%%%%%%%%%%%%%%%%%%%%%%%%%%%%%%%%%%%%%%%%%%%%%%%%%%%%%%%%%%%%%%%%
\chapter{Simulation}
\label{chapter:hps:simulation}
%%%%%%%%%%%%%%%%%%%%%%%%%%%%%%%%%%%%%%%%%%%%%%%%%%%%%%%%%%%%%%%%%%%%%%%%%%%%%%%%

Proper study of the iDM signal model necessitates simulating how the HPS detector
would respond if interactions originating from such a process were to occur.

Signal event production flow
\begin{enumerate}
    \item {\sc MadGraph/MadEvent} generation of prompt iDM events
    \item Displacement of decay products in a \emph{uniform} distribution
    \item Simulation of detector response with \textsc{Geant}4 and \texttt{slic}
    \item Emulation of readout
    \item Normal reconstruction pipeline that data retrieved from the detector
          also goes through
\end{enumerate}

Moreover, simulated samples of standard data allows us to study how the detector
responds in the absence of such a signal process. These samples are produced in
a similar way as iDM signal - just with a different original generation step
initializing the event (and inserting the physical displacement distribution
instead of a uniform one).

%%%%%%%%%%%%%%%%%%%%%%%%%%%%%%%%%%%%%%%%%%%%%%%%%%%%%%%%%%%%%%%%%%%%%%%%%%%%%%%%
