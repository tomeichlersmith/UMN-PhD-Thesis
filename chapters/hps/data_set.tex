\chapter{Data Set}
\label{chapter:hps:dataset}

Proper search for \ac{dm} signals within the \ac{hps} detector includes both collection
of data with the \ac{hps} detetor and simulation of how this apparatus responds to the
physics of specific processes.
This chapter is focused on detailing the origin of these samples.
The collected data can be characterized by the known inputs from \ac{cebaf}
and the run time.
The simulation, as one might expect, is a complicated multi-stage procedure in order
to appropriately reflect the intricacies of the real data.

\section{Collected Data} \label{sec:hps:data}
The data used in this study was collected over a series of 82 data collection runs
within 2016 yielding a total of \todo[lookup]{lookup how much total data was collected in 2016}
days of continuous beam.
The full luminosity of this data is estimated to be $10.7~\text{pb}^{-1}$.
As alluded to in \cref{sec:hps-ecal}, the collected data was triggered in order
to focus the sample on specific data of interest.
The specifics of this trigger are important for simulation as well as understanding
the accessibility of different models from this dataset.
\todo[lookup]{details of 2016 pair-wise trigger should be copied in for reference.}

\section{Simulation} \label{sec:hps:sim}
As mentioned the simulation goes through many steps in order to account
for the different physics processes that are of interest in a realistic fashion.
In general, these steps are
\begin{enumerate}
	\item Generation -- using a tool like {\sc MadGraph/MadEvent}\todo[citation]{MG/ME citation}
	to generate specific events from Feynman diagrams
	\item Displacement -- if the sample expects to have the decay products be displaced
	(for example in the SIMP signal process), displace these decay products
	with a \emph{uniform} distribution of decay lengths to allow for re-weighting.
	\item Simulation -- simulate the detector response with \textsc{Geant4}\cite{geant4}
	\item Emulation -- emulate the readout electronics and triggering mechanism
	of collected data
\end{enumerate}
After this emulation stage, we can treat the simulation the same as the collected data,
applying the reconstruction and further analysis manipulatiuons and selections.

Moreover, simulated samples of standard data allows us to study how the detector
responds in the absence of such a signal process. These samples are produced in
a similar way as iDM signal - just with a different original generation step
initializing the event (and inserting the physical displacement distribution
instead of a uniform one).

\section{Reconstruction}
\todo[lookup]{Brief overview of how events are reconstructed,
especially with regard to tracks and vertices.}

\section{Analysis Pre-Selection}
The final stage that all events go through is a rudimentary pre-selection
which simplifies the resulting shape of the data in the event such that final
analysis is not as complicated.
Specifically, the pre-selection for this analysis is requiring exactly one vertex
to be reconstructed within the event.
This requirement naturally disposes of events which are cluttered with particles
from other beam arrivals (thus causing more than one vertex to be reconstructed)
and events which do not have a electron-positron pair within acceptance
(thus causing no vertices to be reconstructed).
\cref{fig:n-vertex-pre-selection} shows the distribution of number of vertices
within an event for a variety of samples.

\begin{figure}
	\centering
	\includegraphics[width=0.5\textwidth]{figures/hps/dataset/n-vertex-pre-selection-mc-only.pdf}
	\caption{Number of vertices reconstructed within the simulation samples.}
	\label{fig:n-vertex-pre-selection}
\end{figure}