%%%%%%%%%%%%%%%%%%%%%%%%%%%%%%%%%%%%%%%%%%%%%%%%%%%%%%%%%%%%%%%%%%%%%%%%%%%%%%%%
%science.tex: Chapter on DM physics:
%%%%%%%%%%%%%%%%%%%%%%%%%%%%%%%%%%%%%%%%%%%%%%%%%%%%%%%%%%%%%%%%%%%%%%%%%%%%%%%%
\chapter{Dark Matter}
\label{chapter:dm}
%%%%%%%%%%%%%%%%%%%%%%%%%%%%%%%%%%%%%%%%%%%%%%%%%%%%%%%%%%%%%%%%%%%%%%%%%%%%%%%%

Many physicists throughout history have been puzzled by new phenomena, but the
confusion surrounding dark matter -- estimated to be roughly 85\% of the total
matter in the universe -- is surely one of the biggest puzzles. While many
astronomical observations at various scales have confirmed the existence of
dark matter, we have yet seen any observations of its particle interacting
with our detectors. This has ruled out many models for dark matter's particle
nature, but there are still many more available which can both explain the
current observations of dark matter gravitational and cosmological effects
while skirt the limitations defined by the \emph{lack} of observation in
particle experiments.

The theoretical possibilities explaining dark matter are broad \cite{darksectors-2016}.

Trickle through threory of dark matter until reaching Pseudo-Dirac iDM of which
this analysis is studying.

\begin{itemize}
    \item Astronomy observations show lots of measurements
          \begin{itemize}
              \item galactic rotation curves
              \item gravitational lensing
              \item CMB and Baryonic Acoustic Oscilations
              \item Typ1a Supernovae and Big Bang Nucleosynthesis
          \end{itemize}
    \item Can conclude pretty comfortably that DM exists
    \item What is DM's particle nature? Requirements:
          \begin{itemize}
              \item No EM observations $\rightarrow$ Dark
              \item Measurements across time agree $\rightarrow$ Long Living
              \item Measurements across space agree $\rightarrow$ can be observed here
              \item Similar density as SM $\rightarrow$ some interaction (even if weak) that connects their origins (thermal relic)
              \item Universal density must be allowed by model
          \end{itemize}
    \item Whole lot of options
          \begin{itemize}
              \item Thermal relic reminder (i.e. how mass gets connected to cross section due to universal density constraint)
              \item Massive field loop causes dark and standard photon to mix
              \item Details of model left to higher energies (EFT) $\rightarrow$ summarize it as a mixing parameter $\epsilon$
              \item But what happens to the dark photon after that? Could stay in "dark sector" (invisible decay) or return to standard particles (visible decay)
              \item HPS focused on visible decays where the produced electron-positron pair's kinematics may rely on inner-workings of dark sector
              \item Categorize dark sector models ("Vanila", SIMPs, iDM, others...)
          \end{itemize}
    \item Pseudo-Dirac iDM
\end{itemize}

%%%%%%%%%%%%%%%%%%%%%%%%%%%%%%%%%%%%%%%%%%%%%%%%%%%%%%%%%%%%%%%%%%%%%%%%%%%%%%%%
