%%%%%%%%%%%%%%%%%%%%%%%%%%%%%%%%%%%%%%%%%%%%%%%%%%%%%%%%%%%%%%%%%%%%%%%%%%%%%%%%
% experiment.tex: Chapter describing the experiment
%%%%%%%%%%%%%%%%%%%%%%%%%%%%%%%%%%%%%%%%%%%%%%%%%%%%%%%%%%%%%%%%%%%%%%%%%%%%%%%%
\chapter{Light Dark Matter eXperiment}
\label{chapter:ldmx:experiment}

\ac{ldmx} is a proposed fixed-target experiment aiming to definitively explore
the light dark matter phase space. Even as a proposed experiment, it has a detailed
plan for construction, a beam already in construction, and well established connections
with current technologies used within \ac{hep}. While \ac{ldmx} is not yet built,
it has a well formulated simulation infrastructure that can realistically model
how the detector design responds to various types of interactions happening within it.

\section{The Beam Line}
\begin{todoenv}
  Confirm correctness of this section
\end{todoenv}
\ac{ldmx} is situated at the end of the Linear Accelerator (Linac) at SLAC National Accelerator
Laboratory. The SLAC Linac can provide high energy, high rate, and low intensity electron beams for
the various experiments it hosts. Specifically, the Linac Coherent Light Source (LCLS) is
used to guide the beam (of a certain energy) towards the experimental hall - the upgraded
phase of LCLS (LCLS II \cite{lcls-ii}) is currently under construction and is what will
be used for running with \ac{ldmx}.

The experiment is hosted in End Station A (ESA) at SLAC which requires an additional upgrade to the
Linac in order to recieve its beam. LESA (Linac to ESA) \cite{lesa-design} is also currently being
constructed and will be ready for test beam in early 2025. Part of the infrastructure that
transfers the beam to ESA (Sector 30 Transfer Line -- S30XL) is already constructed and LDMX
components are expected to participate in a test beam run in summer 2024.

\section{Detector Design}
LDMX is a missing momentum experiment and its design is focused on measuring \emph{both} the
incoming and outgoing momenta of charged particles interacting with a thin target. This design has
led to four subsystems each with specialized roles.
\begin{enumerate}
  \item \textbf{Trigger Scintillator} Count the number of electrons incident on the target in time to make a trigger decision.
  \item \textbf{Tracker} Measure charged particle momenta both before (``Tagger'') and after (``Recoil'') the target.
  \item \textbf{Electromagnetic Calorimeter} (\ac{ecal}) Measure the total energy of electrons, positrons, and photons.
  \item \textbf{Hadronic Calorimeter} (\ac{hcal}) Veto additional particles difficult for other subsystems to measured (muons, pions, hadrons,...).
\end{enumerate}
\cref{fig:ldmx-det} displays these subsystems in a diagram along with a representation of
a dark brem interaction occurrring within the target.

The Trigger Scintillator (yellow-orange in \cref{fig:ldmx-det}) is made of layers of vertically
segmented bars of plastic scintillator. These layers are arranged in pairs where the layers within
each pair are offset from one another to cover any gaps between the bars. These bars are readout in
time to be used within a trigger decision.\todo{What is a ``trigger'' and how do these bars help?}

The Tracker (purple in \cref{fig:ldmx-det}) is a thin silicon strip detector modeled after the
\ac{hps} tracker. These silicon strips are also arranged in layer pairs where one layer is angled
slightly askew relative to the other in the pair to enable reconstruction of three dimensional hit
locations. The part of the tracker upstream of the target (to the left in \cref{fig:ldmx-det}) is
named the ``tagger'' since its purpose is to measure the incident electrons' momenta, rejecting
electrons' with momentum below $30\%$ of the expected beam momentum. The tagger is situated within
the bulk of the magnetic field enabling highly precise measurement of this incident momentum. The
other part of the tracker located downstream of the target (to the right in \cref{fig:ldmx-det}) is
named the ``recoil'' tracker since its job is to measure the momenta of all charged particles
recoiling from interactions within the target. While it is not located within the magnet volume, it
is still situated within the fringe field, allowing functional momentum resolution.

The \ac{hcal} (green in \cref{fig:ldmx-det}) is a sampling calorimeter made up of alternating
layers of steel absorber and plastic scintillator bars. The \ac{hcal} is further subdivided into
the ``side'' \ac{hcal} which is situated around the \ac{ecal} and the ``back'' \ac{hcal}
downstream of the \ac{ecal}. The back \ac{hcal} has the orientation of the scintillator bars
alternate between vertical and horizontal so that clusters and tracks can have three-dimensional
coordinates more preceisely identified.

The \ac{ecal} (blue in \cref{fig:ldmx-det}), as a primary volume of interest within the analysis
discussed here, is given its own section below \cref{sec:ldmx:ecal}.

\begin{figure}
  \centering
  \includegraphics[width=0.9\textwidth]{figures/ldmx/experiment/detector.png}
  \caption{
    Diagram of LDMX detector apparatus with a representation of a signal event where
    a dark brem occurs within the target. Diagram is not to scale. Credit to Christian Herwig
    for original development of diagram.
  }
  \label{fig:ldmx-det}
\end{figure}

\begin{figure}
  \centering
  \includegraphics[width=0.9\textwidth]{figures/ldmx/experiment/LDMX_FOA_CLOSE.PNG}
  \caption{
    Rendering of LDMX detector apparatus focusing on tracker, target, and ECal.
    The magnet would fully encompass the tracker, target and trigger scintillator.
  }
  \label{fig:ldmx-render}
\end{figure}

\begin{figure}
  \centering
  \includegraphics[width=\textwidth]{figures/ldmx/experiment/reaction_staircase_with_designDrivers.pdf}
  \caption{
    Diagram showing relative rates of background processes within LDMX along with
    how they motivate various aspects of the design. $\slash{E}$ stands for ``missing''
    energy or energy that is ``lost'' to neutrinos that are extremely unlikely to be
    detectable within LDMX.
  }
  \label{fig:ldmx-bkgd-staircase}
\end{figure}

\section{\ac{ecal}}
\label{sec:ldmx:ecal}

More detail about \ac{ecal} since it relates to the ME search later.

\begin{figure}
  \centering
  \includegraphics[width=0.9\textwidth]{figures/ldmx/experiment/ecal.pdf}
  \caption{
    Diagram of LDMX \ac{ecal} construction showing the longitudinal segmentation
    (top and bottom right) and the transverse segmentation (bottom left).
    Credit to Joe Muse.
  }
  \label{fig:ldmx-ecal}
\end{figure}

\section{Missing Momentum \ac{dm} Search}

Description of MM search using the thin target.

\section{Missing Energy \ac{dm} Search} One of the primary strengths of the LDMX detector design is its ability to use the tagging
and recoil tracker system to reject a large number of background events by separating a nominal
beam with non-standard energy loss from a nominal beam with standard energy loss or even low-energy
beam. Moreover, the tagging and recoil tracker system gives LDMX the potential to further suppress
backgrounds or potentially study DM properties by studying the transverse momenta of electrons
recoiling from dark-bremsstrahlung candidate events.

While this design is optimal for a large number of EoT, the strategy has some limitations for a low
EoT data run. To limit multiple scattering which ruins momentum measurements, the baseline detector
configuration requires a thin ($\approx 0.1 \,\mathrm{X}_\circ$) target. In the early stages of
LDMX, when the total EoT will be lower, a different analysis strategy and detector configuration
may be optimal to probe the largest amount of the $y$--$m_\chi$ phase space. The alternative
strategy would ignore the dedicated target inside the tracker volume and instead use the ECal as an
active target. Using the ECal as Target (EaT) for this early-running phase increases the potential
number of dark-bremsstrahlung events and, as long as backgrounds can be suppressed relative to
signal, allows a stronger search capability for a fixed EoT.

%An initial study of the EaT analysis channel is the primary focus of \cref{part:ldmx} of this thesis. Comparing various signal hypotheses against the pesky photon-nuclear and electron-nuclear background (here called ``Enriched Nuclear") shows that a simple missing energy cut in the ECal and a maximum activity cut in the HCal eliminates the background events to less than one event out of $10^{13}$ EoT (Table \ref{tab:eat_cutflow}). While LDMX has many other handles for eliminating these Enriched Nuclear events, these rudimentary cuts already show that LDMX has the capability to reach into new DM phase space with only a small fraction of events in the first phase of running (Figure \ref{fig:eat_reach}).

%%%%%%%%%%%%%%%%%%%%%%%%%%%%%%%%%%%%%%%%%%%%%%%%%%%%%%%%%%%%%%%%%%%%%%%%%%%%%%%%
