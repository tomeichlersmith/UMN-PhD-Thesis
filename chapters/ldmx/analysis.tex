%%%%%%%%%%%%%%%%%%%%%%%%%%%%%%%%%%%%%%%%%%%%%%%%%%%%%%%%%%%%%%%%%%%%%%%%%%%%%%%%
% After generating data, what we do to select signal and prepare for search
%%%%%%%%%%%%%%%%%%%%%%%%%%%%%%%%%%%%%%%%%%%%%%%%%%%%%%%%%%%%%%%%%%%%%%%%%%%%%%%%
\chapter{Analysis}
\label{chapter:ldmx:analysis}
%%%%%%%%%%%%%%%%%%%%%%%%%%%%%%%%%%%%%%%%%%%%%%%%%%%%%%%%%%%%%%%%%%%%%%%%%%%%%%%%

The \ac{eat} analysis channel for \ac{ldmx} has two primary purposes which can
be separated by the timeline over which they are relevant.
\begin{enumerate}
	\item \textbf{Short Term}: In early running, when the number of \ac{eot} is
	      relatively small, the nominal \ac{mm} analysis will not have obtained
	      significant reach into new \ac{dm} phase space (yet). The \ac{eat} channel
	      serves here as a way to obtain world-leading sensitivity early in the lifetime
	      of \ac{ldmx} and give the collaboration a first look at the data the apparatus
	      has collected.
	\item \textbf{Long Term}: As \ac{ldmx} collects data, the \ac{mm}
	      analysis enters into unexplored phase space and serves as a better discovery
	      mechanism due to its access to the Tagger and Recoil trackers. The \ac{eat}
	      channel, while struggling to suppress complicated backgrounds with relatively
	      limited analysis handles, can operate ``orthogonally'' in the collected data
	      since its primary selection (an approximately beam-energy electron passing
	      through the Recoil tracker) is inverted relative to the \ac{mm} analysis
	      (the electron passing through the Recoil tracker has significantly less
	      energy than the beam).
\end{enumerate}
As a first investigation of the \ac{eat} analysis channel, this work is focused on
the first (short term) purpose. In this regard, we target an \ac{eot} that is reasonable
to accomplish early in the running of \ac{ldmx} and avoids particularly intricate backgrounds.
\num{1e13} \ac{eot} fits these requirements by avoiding the charged current production
of neutrinos and represents $\sim\qty{10}{percent}$ of the first full \ac{ldmx} dataset.
obtainable within approximately a week of beam time\footnote{
	Assuming the \ac{ldmx} detector apparatus and beam delivery is operating according to specifications, we expect the beam to be delivered on a frequency of \qty{37.5}{\mega\hertz} with a duty cycle of $\approx$\num{0.5}.
	$\qty{37.5}{\mega\hertz}\times0.5\times\qty{1}{week}\approx\num{1.1e13}$~\ac{eot}.
}.
Since \ac{eat} is expected to be the \emph{first} physics
analysis on \ac{ldmx} data, we want a \emph{simple} and \emph{robust} analysis that can withstand
the test of time and the complexities of real data.

With these design goals in mind, a simple ``cut-and-count'' analysis has been developed.
The simplicity of this analysis is one of its strengths, enabling it to be applicable
despite potential surprises arising from first encounters with real data.
The bulk of time and effort on this first investigation was focused on making this
investigation \emph{possible} via the introduction of midshower process filtering
and a dark bremsstrahlung simulatin process described in \cref{chapter:ldmx:simulation}.

\section{Trigger}
Use same on-line trigger as MM analysis.

Re-apply trigger threshold to entire ecal energy sum off-line (same as MM analysis)

\section{Cut and Count}
List cuts for different beams, provide final background yields and signal efficiencies.

\section{Reach}
Display reach of ME analysis with different background hypotheses and signal efficiencies
motivated by findings above.

%%%%%%%%%%%%%%%%%%%%%%%%%%%%%%%%%%%%%%%%%%%%%%%%%%%%%%%%%%%%%%%%%%%%%%%%%%%%%}}}
