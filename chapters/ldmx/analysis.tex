%%%%%%%%%%%%%%%%%%%%%%%%%%%%%%%%%%%%%%%%%%%%%%%%%%%%%%%%%%%%%%%%%%%%%%%%%%%%%%%%
% After generating data, what we do to select signal and prepare for search
%%%%%%%%%%%%%%%%%%%%%%%%%%%%%%%%%%%%%%%%%%%%%%%%%%%%%%%%%%%%%%%%%%%%%%%%%%%%%%%%
\chapter{Analysis}
\label{chapter:ldmx:analysis}
%%%%%%%%%%%%%%%%%%%%%%%%%%%%%%%%%%%%%%%%%%%%%%%%%%%%%%%%%%%%%%%%%%%%%%%%%%%%%%%%

General analysis purpose: we want a \emph{simple} and \emph{robust} analysis that can withstand the test of time and complexity of real data.

Point out that we check two beam energies since there is some uncertainty about which
beam energy LDMX will be able to see first.

\section{Trigger}
Use same on-line trigger as MM analysis.

Re-apply trigger threshold to entire ecal energy sum off-line (same as MM analysis)

\section{Cut and Count}
List cuts for different beams, provide final background yields and signal efficiencies.

\section{Reach}
Display reach of ME analysis with different background hypotheses and signal efficiencies
motivated by findings above.

%%%%%%%%%%%%%%%%%%%%%%%%%%%%%%%%%%%%%%%%%%%%%%%%%%%%%%%%%%%%%%%%%%%%%%%%%%%%%}}}
