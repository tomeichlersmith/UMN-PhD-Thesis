%%%%%%%%%%%%%%%%%%%%%%%%%%%%%%%%%%%%%%%%%%%%%%%%%%%%%%%%%%%%%%%%%%%%%%%%%%%%%%%%
% simulation.tex: Chapter on MC production:
%%%%%%%%%%%%%%%%%%%%%%%%%%%%%%%%%%%%%%%%%%%%%%%%%%%%%%%%%%%%%%%%%%%%%%%%%%%%%%%%
\chapter{Simulation}
\label{chapter:ldmx:simulation}
%%%%%%%%%%%%%%%%%%%%%%%%%%%%%%%%%%%%%%%%%%%%%%%%%%%%%%%%%%%%%%%%%%%%%%%%%%%%%%%%

General description of LDMX software and how it constructs a data processing
pipeline. Geant4, MadGraph/MadEvent, electronics emulation, and reconstruction.

\section{Standard Processes}
``Backgrounds'' -- version of Geant4 and how it has been configured (broadly)

\subsection{Biasing and Filtering}
Biasing and filtering methodology, configuration for EaT-specific backgrounds.

\subsection{Validation}
Validation of this methodology.

\section{\gls{dm} Signal}
The particular signal process this analysis channel is looking for is the
production of a dark photon followed by an \emph{invisible} decay. In this
regime, what happens to the dark photon after it is produced is irrelevant
to the analysis since both it and its products are not observable by our
detector.

With this focus in mind, we developed a dark bremsstrahlung simulation method
that allows for the visible particle (the recoiling lepton) to be distributed
according to a full matrix element calculation (via MG/ME) while the incident
particle can have varied energy and be handled by Geant4 directly. This novel
simulation technique allows for the dark bremsstrahlung process to be treated
(from Geant4's perspective) on the same footing as the background processes
while maintaining the precision of a matrix-calculator method.

\subsection{G4DarkBreM}
Description of scaling as well as citation to paper.

\subsection{MadGraph/MadEvent}
Detail on MG/ME workspace used to generate DB libraries.

\subsection{Characterization}
Describe how these samples "look"?

%%%%%%%%%%%%%%%%%%%%%%%%%%%%%%%%%%%%%%%%%%%%%%%%%%%%%%%%%%%%%%%%%%%%%%%%%%%%%%%%
