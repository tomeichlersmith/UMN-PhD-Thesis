Astrophysical evidence strongly indicates the presence of particulate Dark Matter (DM) within our universe; however, the specific particle nature of DM is still unknown. The wide variety of possible DM particles produces a similar range of experiments focused on probing these different categories of possible DM. This talk describes two experiments taking different approaches to search for Light DM residing in the 1MeV-1GeV mass range being produced by electron interactions. The Light Dark Matter eXperiment (LDMX) is a proposed fixed target experiment designed for a missing momentum search with an additional, orthogonal missing energy search channel described in this talk. The Heavy Photon Search experiment (HPS) is another fixed target experiment designed for a displaced vertex search with distances of O(10cm) which are not probed by longer baseline experiments. Specifically, a search in HPS data for a specific Light DM model with a strongly-coupled dark sector enabling a higher expected production rate while keeping the characteristic decay length within HPS's acceptance is also presented.
